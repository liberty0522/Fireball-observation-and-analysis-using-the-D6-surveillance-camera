\chapter{Background}

While the Earth's atmosphere is thick enough to cause more objects that pass through it, some can and will make it through. Theses are the objects of most importance to those observing fireballs, meteoroids, and meteors. This chapter will provide a more in-depth look at what classifies an object as a fireball, how these fireballs are detected both by other systems around the world and the D6, and discuss why and how the detected images are analyzed.

\section{What Exactly are Fireballs?}

The first thing to take into account when discussing meteoroids, meteors, fireballs, and even asteroids, is that they are all different and shouldn't be used interchangeably. 
An asteroid is not a meteoroid, a meteoroid is not a meteor, a fireball is a classification of meteor, and a bolide is a fireball that blows up in the atmosphere. 
This section will go into a little more detail regarding the various definitions of these space objects, 

Asteroids are meteoroids that are greater than 1 meter wide, as can be seen in figure 3, they tend to be concentrated outside of the Earth's orbit, as can be seen in the third figure.
As can be seen from this concentration, and from what has been determined by the scientific community, large asteroid impacts tend to occur varely rarely, with "an automobile sized asteroid" entering the atmosphere about once a year \cite{atkinson_2018}. 
Meteoroids however, are the little siblings of asteroids, and are of just as much interest.

\begin{figure}
    \centering
    \includegraphics[width=12cm]{Solar-System-Asteroids.png}
    \centering
    \caption{Image of asteroid plots in our Solar System, courtesy of Asterank, as can be seen the asteroids mainly orbit outside the orbit of the Earth, though some do merge in and out of the Earth's orbit.}
    \label{Figure 3}
\end{figure}

Meteoroids are any bits and bobs of cosmic space debris that are up to 1 meter in width, from dust and small pieces of rock blasted off from asteroid collisions. The only true difference between an asteroid and a meteoroid is that one is large, while one is usually much smaller, with the meteroid being frequently only millimeters in size \cite{atkinson_2018}. 
When both asteroids and meteoroids hit the atmosphere and begin leaving a streak of fire across the sky, they are then referred to as meteors. 
While meteors are harmless most of the time, they can turn into bolides, also known as fireballs \cite{atkinson_2018}. 

Fireballs, or bolides, will be the main focus of the project, as they are what are being observed for by the D6 and the other detection systems that the D6 will be compared to when determining it's effectiveness. 
Fireballs are meteors that have especially bright trails, and tend to either burn out in the atmosphere, explode, or leave behind a meteorite like the one that occurred over Spain, Portugal, and France, with a meteorite being left behind in northern Palencia \cite{Villabeto}. 
In the next segment a more in depth look will be taken at the different methods used to detect the fireballs, as well as some of positives and negatives for each of these detection methods.

\section{Methods of Detection}

One of the major detection networks is called CAMS, or, Cameras for Allsky Meteor Surveillance. 
The goal of this NASA funded project is to validating unconfirmed meteor showers by measuring velocity vectors and times of arrival and from that information, determining if various detected meteoroids and fireballs originated from the same group passing through the Earth's orbit \cite{jenniskens}. 
The stations for the detection network contain a box within which twenty Watec 902 H2 Ultimate video cameras, an interesting feature of this camera setup is that it comes with a sun shade, which can be used to protect the setup from bad weather. 
The sun shade also helps protect the sensitive cameras from direct sunlight during daytime hours \cite{jenniskens}. 
An important feature of this setup to note that is the camera also has a battery mounted nearby to support it, as well as the linux surveillance servers required to process and store the data from each of the cameras.As can be seen in Figures 4 and 5, the total setup is fairly bulky, although the use of an array of cameras for comparison will allow for fairly good detail in any capture images, as well as the capability to determine the velocity of a fireball as it moves across the array.

\begin{figure}
    \centering
    \includegraphics[scale=1.5]{CAMS-Server-Setup.png}
    \caption{Image of the Server setup for the CAMS network}
    \label{Figure 4}
\end{figure}

\begin{figure}
    \centering
    \includegraphics[scale=1.5]{CAMS-Camera-Setup.png}
    \caption{Image of the Camera setup for the CAMS network, from}
    \label{Figure 5}
\end{figure}


\section{Previous Work Done on the Project}

How it's analyzed, discuss Luke's luminosity code, PJ's work with observational area, how we want to measure the flux based off observational energy.